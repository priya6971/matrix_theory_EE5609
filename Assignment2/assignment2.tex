\documentclass[journal,12pt,twocolumn]{IEEEtran}

\usepackage{setspace}
\usepackage{gensymb}


\singlespacing

\usepackage[cmex10]{amsmath}
%\usepackage{amsthm}
%\interdisplaylinepenalty=2500
%\savesymbol{iint}
%\usepackage{txfonts}
%\restoresymbol{TXF}{iint}
%\usepackage{wasysym}
\usepackage{amsthm}

\usepackage{mathrsfs}
\usepackage{txfonts}
\usepackage{stfloats}
\usepackage{bm}
\usepackage{cite}
\usepackage{cases}
\usepackage{subfig}

\usepackage{longtable}
\usepackage{multirow}

\usepackage{enumitem}
\usepackage{mathtools}
\usepackage{steinmetz}
\usepackage{tikz}
\usepackage{circuitikz}
\usepackage{verbatim}
\usepackage{tfrupee}
\usepackage[breaklinks=true]{hyperref}

\usepackage{tkz-euclide} %loads TikZ and tkz-base

\usetikzlibrary{calc,math}
\usepackage{listings}
    \usepackage{color}                                          
    \usepackage{array}                                          
    \usepackage{longtable}                                      
    \usepackage{calc}                                          
    \usepackage{multirow}                                      
    \usepackage{hhline}                                        
    \usepackage{ifthen}
    \usepackage{lscape}    
\usepackage{multicol}
\usepackage{chngcntr}

\DeclareMathOperator*{\Res}{Res}

\renewcommand\thesection{\arabic{section}}
\renewcommand\thesubsection{\thesection.\arabic{subsection}}
\renewcommand\thesubsubsection{\thesubsection.\arabic{subsubsection}}

\renewcommand\thesectiondis{\arabic{section}}
\renewcommand\thesubsectiondis{\thesectiondis.\arabic{subsection}}
\renewcommand\thesubsubsectiondis{\thesubsectiondis.\arabic{subsubsection}}

\hyphenation{op-tical net-works semi-conduc-tor}
\def\inputGnumericTable{}                                 %%

\lstset{
%language=C,
frame=single,
breaklines=true,
columns=fullflexible
}

\begin{document}

\newtheorem{theorem}{Theorem}[section]
\newtheorem{problem}{Problem}
\newtheorem{proposition}{Proposition}[section]
\newtheorem{lemma}{Lemma}[section]
\newtheorem{corollary}[theorem]{Corollary}
\newtheorem{example}{Example}[section]
\newtheorem{definition}[problem]{Definition}

\newcommand{\BEQA}{\begin{eqnarray}}
\newcommand{\EEQA}{\end{eqnarray}}
\newcommand{\define}{\stackrel{\triangle}{=}}
\bibliographystyle{IEEEtran}
\providecommand{\mbf}{\mathbf}
\providecommand{\pr}[1]{\ensuremath{\Pr\left(#1\right)}}
\providecommand{\qfunc}[1]{\ensuremath{Q\left(#1\right)}}
\providecommand{\sbrak}[1]{\ensuremath{{}\left[#1\right]}}
\providecommand{\lsbrak}[1]{\ensuremath{{}\left[#1\right.}}
\providecommand{\rsbrak}[1]{\ensuremath{{}\left.#1\right]}}
\providecommand{\brak}[1]{\ensuremath{\left(#1\right)}}
\providecommand{\lbrak}[1]{\ensuremath{\left(#1\right.}}
\providecommand{\rbrak}[1]{\ensuremath{\left.#1\right)}}
\providecommand{\cbrak}[1]{\ensuremath{\left\{#1\right\}}}
\providecommand{\lcbrak}[1]{\ensuremath{\left\{#1\right.}}
\providecommand{\rcbrak}[1]{\ensuremath{\left.#1\right\}}}
\theoremstyle{remark}
\newtheorem{rem}{Remark}
\newcommand{\sgn}{\mathop{\mathrm{sgn}}}
\providecommand{\abs}[1]{\left\vert#1\right\vert}
\providecommand{\res}[1]{\Res\displaylimits_{#1}}
\providecommand{\norm}[1]{\left\lVert#1\right\rVert}
%\providecommand{\norm}[1]{\lVert#1\rVert}
\providecommand{\mtx}[1]{\mathbf{#1}}
\providecommand{\mean}[1]{E\left[ #1 \right]}
\providecommand{\fourier}{\overset{\mathcal{F}}{ \rightleftharpoons}}
%\providecommand{\hilbert}{\overset{\mathcal{H}}{ \rightleftharpoons}}
\providecommand{\system}{\overset{\mathcal{H}}{ \longleftrightarrow}}
%\newcommand{\solution}[2]{\textbf{Solution:}{#1}}
\newcommand{\solution}{\noindent \textbf{Solution: }}
\newcommand{\cosec}{\,\text{cosec}\,}
\providecommand{\dec}[2]{\ensuremath{\overset{#1}{\underset{#2}{\gtrless}}}}
\newcommand{\myvec}[1]{\ensuremath{\begin{pmatrix}#1\end{pmatrix}}}
\newcommand{\mydet}[1]{\ensuremath{\begin{vmatrix}#1\end{vmatrix}}}
\numberwithin{equation}{subsection}
\makeatletter
\@addtoreset{figure}{problem}
\makeatother
\let\StandardTheFigure\thefigure
\let\vec\mathbf
\renewcommand{\thefigure}{\theproblem}
\def\putbox#1#2#3{\makebox[0in][l]{\makebox[#1][l]{}\raisebox{\baselineskip}[0in][0in]{\raisebox{#2}[0in][0in]{#3}}}}
     \def\rightbox#1{\makebox[0in][r]{#1}}
     \def\centbox#1{\makebox[0in]{#1}}
     \def\topbox#1{\raisebox{-\baselineskip}[0in][0in]{#1}}
     \def\midbox#1{\raisebox{-0.5\baselineskip}[0in][0in]{#1}}
\vspace{3cm}
\title{Assignment 2}
\author{Priya Bhatia}
\maketitle
\newpage
%\tableofcontents
\bigskip
\renewcommand{\thefigure}{\theenumi}
\renewcommand{\thetable}{\theenumi}
\begin{abstract}
This document balances the Chemical Equation.
\end{abstract}
Download all python codes from
\begin{lstlisting}
https://github.com/priya6971/matrix_theory_EE5609/tree/master/Assignment2/code
\end{lstlisting}
%
and latex-tikz codes from
%
\begin{lstlisting}
https://github.com/priya6971/matrix_theory_EE5609/tree/master/Assignment2
\end{lstlisting}
%
\section{Problem}
Write the balanced chemical equations for the below reaction.
\begin{align}\label{1}
    Ca(OH)_2 + CO_2 \xrightarrow{} CaCO_3 + H_2O
\end{align}
\section{Explanation}
Let the balanced version of \eqref{1} be:-
\begin{align}\label{2}
x_1Ca(OH)_2 + x_2CO_2 \xrightarrow{} x_3CaCO_3 + x_4H_2O
\end{align}
which results in the following equations:
\begin{equation}
 \begin{aligned}
    (x_1-x_3)Ca=0\\
    (2x_1+2x_2-3x_3-x_4)O=0\\
    (2x_1-2x_4)H=0\\
    (x_2-x_3)C=0
 \end{aligned}
\end{equation}
which can be expressed as:-
\begin{equation}
 \begin{aligned}
    1.x_1 + 0.x_2 - 1.x_3 + 0.x_4=0\\
    2.x_1 + 2.x_2 - 3.x_3 - 1.x_4=0\\
    2.x_1 + 0.x_2 + 0.x_3 - 2.x4=0\\
    0.x_1 + 1.x_2 - 1.x_3 + 0.x_4=0
 \end{aligned}
\end{equation}
resulting in the matrix equation:-
\begin{align}\label{3}
    \myvec{1 & 0 & -1 & 0\\
    2 & 2 & -3 & -1\\
    2 & 0 & 0 & -2\\
    0 & 1 & -1 & 0}\vec{x}=\vec{0}
\end{align}
where
\begin{align}
    \vec{x}=\myvec{x_1\\x_2\\x_3\\x_4}
\end{align}
\section{Solution}
Now by applying the Row Reduction Method in the equation \eqref{3} we can easily determine the value of the coefficients of the Chemical Equation:

Equation \eqref{3} can be reduced as follows:
\begin{align}
   \myvec{1 & 0 & -1 & 0\\
    2 & 2 & -3 & -1\\
    2 & 0 & 0 & -2\\
    0 & 1 & -1 & 0}\xleftrightarrow[]{R2\leftarrow R2-2R1}\myvec{
1 & 0 & -1 & 0\\
0 & 2 & -1 & -1\\
2 & 0 & 0 & -2\\
0 & 1 & -1 & 0}
\end{align}
\begin{align}
   \myvec{
1 & 0 & -1 & 0\\
0 & 2 & -1 & -1\\
2 & 0 & 0 & -2\\
0 & 1 & -1 & 0}\xleftrightarrow[]{R3\leftarrow R3-2R1} \myvec{
1 & 0 & -1 & 0\\
0 & 2 & -1 & -1\\
0 & 0 & 2 & -2\\
0 & 1 & -1 & 0}
\end{align}
Thus,
\begin{align}
    x_1=x_3,x_2=x_3,x_3=x_4\\
    \vec{x}=\myvec{x_1\\x_2\\x_3\\x_4}=\myvec{1\\1\\1\\1}\label{4}
\end{align}
Then \eqref{2} remains same because the value of $x_1=1$,$x_2=1$,$x_3=1$,$x_4=1$ as shown in equation \eqref{4},
\begin{align}\label{5}
    \boxed{Ca(OH)_2 + CO_2 \xrightarrow{} CaCO_3 + H_2O}
\end{align}
 \eqref{5} is our required balance equation.
\end{document}