\documentclass[journal,12pt,twocolumn]{IEEEtran}

\usepackage{setspace}
\usepackage{gensymb}


\singlespacing

\usepackage[cmex10]{amsmath}
%\usepackage{amsthm}
%\interdisplaylinepenalty=2500
%\savesymbol{iint}
%\usepackage{txfonts}
%\restoresymbol{TXF}{iint}
%\usepackage{wasysym}
\usepackage{amsthm}

\usepackage{mathrsfs}
\usepackage{txfonts}
\usepackage{stfloats}
\usepackage{bm}
\usepackage{cite}
\usepackage{cases}
\usepackage{subfig}

\usepackage{longtable}
\usepackage{multirow}

\usepackage{enumitem}
\usepackage{mathtools}
\usepackage{steinmetz}
\usepackage{tikz}
\usepackage{circuitikz}
\usepackage{verbatim}
\usepackage{tfrupee}
\usepackage[breaklinks=true]{hyperref}

\usepackage{tkz-euclide} %loads TikZ and tkz-base

\usetikzlibrary{calc,math}
\usepackage{listings}
    \usepackage{color}                                          
    \usepackage{array}                                          
    \usepackage{longtable}                                      
    \usepackage{calc}                                          
    \usepackage{multirow}                                      
    \usepackage{hhline}                                        
    \usepackage{ifthen}
    \usepackage{lscape}    
\usepackage{multicol}
\usepackage{chngcntr}

\DeclareMathOperator*{\Res}{Res}

\renewcommand\thesection{\arabic{section}}
\renewcommand\thesubsection{\thesection.\arabic{subsection}}
\renewcommand\thesubsubsection{\thesubsection.\arabic{subsubsection}}

\renewcommand\thesectiondis{\arabic{section}}
\renewcommand\thesubsectiondis{\thesectiondis.\arabic{subsection}}
\renewcommand\thesubsubsectiondis{\thesubsectiondis.\arabic{subsubsection}}

\hyphenation{op-tical net-works semi-conduc-tor}
\def\inputGnumericTable{}                                 %%

\lstset{
%language=C,
frame=single,
breaklines=true,
columns=fullflexible
}

\begin{document}

\newtheorem{theorem}{Theorem}[section]
\newtheorem{problem}{Problem}
\newtheorem{proposition}{Proposition}[section]
\newtheorem{lemma}{Lemma}[section]
\newtheorem{corollary}[theorem]{Corollary}
\newtheorem{example}{Example}[section]
\newtheorem{definition}[problem]{Definition}

\newcommand{\BEQA}{\begin{eqnarray}}
\newcommand{\EEQA}{\end{eqnarray}}
\newcommand{\define}{\stackrel{\triangle}{=}}
\bibliographystyle{IEEEtran}
\providecommand{\mbf}{\mathbf}
\providecommand{\pr}[1]{\ensuremath{\Pr\left(#1\right)}}
\providecommand{\qfunc}[1]{\ensuremath{Q\left(#1\right)}}
\providecommand{\sbrak}[1]{\ensuremath{{}\left[#1\right]}}
\providecommand{\lsbrak}[1]{\ensuremath{{}\left[#1\right.}}
\providecommand{\rsbrak}[1]{\ensuremath{{}\left.#1\right]}}
\providecommand{\brak}[1]{\ensuremath{\left(#1\right)}}
\providecommand{\lbrak}[1]{\ensuremath{\left(#1\right.}}
\providecommand{\rbrak}[1]{\ensuremath{\left.#1\right)}}
\providecommand{\cbrak}[1]{\ensuremath{\left\{#1\right\}}}
\providecommand{\lcbrak}[1]{\ensuremath{\left\{#1\right.}}
\providecommand{\rcbrak}[1]{\ensuremath{\left.#1\right\}}}
\theoremstyle{remark}
\newtheorem{rem}{Remark}
\newcommand{\sgn}{\mathop{\mathrm{sgn}}}
\providecommand{\abs}[1]{\left\vert#1\right\vert}
\providecommand{\res}[1]{\Res\displaylimits_{#1}}
\providecommand{\norm}[1]{\left\lVert#1\right\rVert}
%\providecommand{\norm}[1]{\lVert#1\rVert}
\providecommand{\mtx}[1]{\mathbf{#1}}
\providecommand{\mean}[1]{E\left[ #1 \right]}
\providecommand{\fourier}{\overset{\mathcal{F}}{ \rightleftharpoons}}
%\providecommand{\hilbert}{\overset{\mathcal{H}}{ \rightleftharpoons}}
\providecommand{\system}{\overset{\mathcal{H}}{ \longleftrightarrow}}
%\newcommand{\solution}[2]{\textbf{Solution:}{#1}}
\newcommand{\solution}{\noindent \textbf{Solution: }}
\newcommand{\cosec}{\,\text{cosec}\,}
\providecommand{\dec}[2]{\ensuremath{\overset{#1}{\underset{#2}{\gtrless}}}}
\newcommand{\myvec}[1]{\ensuremath{\begin{pmatrix}#1\end{pmatrix}}}
\newcommand{\mydet}[1]{\ensuremath{\begin{vmatrix}#1\end{vmatrix}}}
\numberwithin{equation}{subsection}
\makeatletter
\@addtoreset{figure}{problem}
\makeatother
\let\StandardTheFigure\thefigure
\let\vec\mathbf
\renewcommand{\thefigure}{\theproblem}
\def\putbox#1#2#3{\makebox[0in][l]{\makebox[#1][l]{}\raisebox{\baselineskip}[0in][0in]{\raisebox{#2}[0in][0in]{#3}}}}
     \def\rightbox#1{\makebox[0in][r]{#1}}
     \def\centbox#1{\makebox[0in]{#1}}
     \def\topbox#1{\raisebox{-\baselineskip}[0in][0in]{#1}}
     \def\midbox#1{\raisebox{-0.5\baselineskip}[0in][0in]{#1}}
\vspace{3cm}
\title{Challenge Problem}
\author{Priya Bhatia}
\maketitle
\newpage
%\tableofcontents
\bigskip
\renewcommand{\thefigure}{\theenumi}
\renewcommand{\thetable}{\theenumi}
\begin{abstract}
This document show that Orthogonal vectors are linearly independent
\end{abstract}
%
Download latex-tikz codes from
%
\begin{lstlisting}
https://github.com/priya6971/matrix_theory_EE5609/tree/master/ChallengeProblem
\end{lstlisting}
%
\section{Problem}
Show that the set of Orthogonal vectors is Linear independent. 
\section{Proof}
\begin{align}
c_1\vec{v_1}+c_2\vec{v_2}+\dots+c_n\vec{v_n}= 0\label{eq1}
\end{align}
We have to show that in \eqref{eq1}, $c_1=0$, $c_2=0$ and so on upto $c_n=0$.
We begin by taking only two orthogonal vectors say $\vec{v_1}$ and $\vec{v_2}$ are the two orthogonal vectors.

And we know that $\vec{v_1}$ and $\vec{v_2}$ are Linearly Independent if and only if the value of $c_1=0$, $c_2=0$ in below equation:
\begin{align}
c_1\vec{v_1}+c_2\vec{v_2} = 0\label{eq2}
\end{align}
To prove this, we can take the dot product of $\vec{v_1}$ on both side in $\vec{v_1}$
\begin{align}
c_1\vec{v_1}\vec{v_1}+c_2\vec{v_1}\vec{v_2} = 0\label{eq3}
\end{align}
Now as $\vec{v_1}$ and $\vec{v_2}$ are orthogonal vectors so dot product $\vec{v_1}$ and $\vec{v_2}$ is 0.
Therefore we get from \eqref{eq3}
\begin{align}
c_1\vec{v_1}\vec{v_1} = 0\label{eq4}
\end{align}
Now $\vec{v_1}$ cannot be zero as $\vec{v_1}$ is from a set of non-zero orthogonal vectors.
Therefore we get $c_1=0$, from \eqref{eq4}.
And simlarly we can proof that the value of $c_2=0$, by taking dot product of vector $\vec{v_2}$ in equation \eqref{eq2}

Thus, orthogonal vectors $\vec{v_1}$ and $\vec{v_2}$ satisfy the condition of linear independence.
\subsection{General Case}
Consider, the expression
\begin{equation}\label{1}
c_1\vec{v_1} + c_2\vec{v_2} + ... + c_n\vec{v_n} = 0
\end{equation}
Take the dot product of \ref{1} with $\vec{v_1}$, we get
\begin{align}
&	c_1\norm{\vec{v_1}}^2 + c_2\vec{v_2}^T\vec{v_1} + ... + c_n\vec{v_n}^T\vec{v_1} = 0 \\
&   c_1\norm{\vec{v_1}}^2 = 0 \quad (\vec{v_i}^T\vec{v_j}=0 \quad \forall i\not=j)\\
& \quad \norm{\vec{v_1}}^2 = 0 \quad \iff \vec{v_1} = 0 
\end{align}
Hence, $c_1 = 0$
Similarly, taking the dot product of \ref{1} with $\vec{v_2}$, ...,$\vec{v_n}$, we find out $c_2 =0, ...,c_n=0$.
Thus, the set of Orthogonal vectors $\vec{v_1},\vec{v_2},...,\vec{v_n}$ is Linear independent.

So, we can proof that if $\vec{v_1} ,\vec{v_2}$ upto $\vec{v_n}$ are Orthogonal vectors that forms an equation \eqref{eq1}.

Then, the value of $c_1=0$, $c_2=0$ and so on upto $c_n=0$.
\end{document}
