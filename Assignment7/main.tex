\documentclass[journal,12pt,twocolumn]{IEEEtran}

\usepackage{setspace}
\usepackage{gensymb}


\singlespacing

\usepackage[cmex10]{amsmath}
%\usepackage{amsthm}
%\interdisplaylinepenalty=2500
%\savesymbol{iint}
%\usepackage{txfonts}
%\restoresymbol{TXF}{iint}
%\usepackage{wasysym}
\usepackage{amsthm}

\usepackage{mathrsfs}
\usepackage{txfonts}
\usepackage{stfloats}
\usepackage{bm}
\usepackage{cite}
\usepackage{cases}
\usepackage{subfig}

\usepackage{longtable}
\usepackage{multirow}

\usepackage{enumitem}
\usepackage{mathtools}
\usepackage{steinmetz}
\usepackage{tikz}
\usepackage{circuitikz}
\usepackage{verbatim}
\usepackage{tfrupee}
\usepackage[breaklinks=true]{hyperref}

\usepackage{tkz-euclide} %loads TikZ and tkz-base

\usetikzlibrary{calc,math}
\usepackage{listings}
    \usepackage{color}                                          
    \usepackage{array}                                          
    \usepackage{longtable}                                      
    \usepackage{calc}                                          
    \usepackage{multirow}                                      
    \usepackage{hhline}                                        
    \usepackage{ifthen}
    \usepackage{lscape}    
\usepackage{multicol}
\usepackage{chngcntr}

\DeclareMathOperator*{\Res}{Res}

\renewcommand\thesection{\arabic{section}}
\renewcommand\thesubsection{\thesection.\arabic{subsection}}
\renewcommand\thesubsubsection{\thesubsection.\arabic{subsubsection}}

\renewcommand\thesectiondis{\arabic{section}}
\renewcommand\thesubsectiondis{\thesectiondis.\arabic{subsection}}
\renewcommand\thesubsubsectiondis{\thesubsectiondis.\arabic{subsubsection}}

\hyphenation{op-tical net-works semi-conduc-tor}
\def\inputGnumericTable{}                                 %%

\lstset{
%language=C,
frame=single,
breaklines=true,
columns=fullflexible
}

\begin{document}

\newtheorem{theorem}{Theorem}[section]
\newtheorem{problem}{Problem}
\newtheorem{proposition}{Proposition}[section]
\newtheorem{lemma}{Lemma}[section]
\newtheorem{corollary}[theorem]{Corollary}
\newtheorem{example}{Example}[section]
\newtheorem{definition}[problem]{Definition}

\newcommand{\BEQA}{\begin{eqnarray}}
\newcommand{\EEQA}{\end{eqnarray}}
\newcommand{\define}{\stackrel{\triangle}{=}}
\bibliographystyle{IEEEtran}
\providecommand{\mbf}{\mathbf}
\providecommand{\pr}[1]{\ensuremath{\Pr\left(#1\right)}}
\providecommand{\qfunc}[1]{\ensuremath{Q\left(#1\right)}}
\providecommand{\sbrak}[1]{\ensuremath{{}\left[#1\right]}}
\providecommand{\lsbrak}[1]{\ensuremath{{}\left[#1\right.}}
\providecommand{\rsbrak}[1]{\ensuremath{{}\left.#1\right]}}
\providecommand{\brak}[1]{\ensuremath{\left(#1\right)}}
\providecommand{\lbrak}[1]{\ensuremath{\left(#1\right.}}
\providecommand{\rbrak}[1]{\ensuremath{\left.#1\right)}}
\providecommand{\cbrak}[1]{\ensuremath{\left\{#1\right\}}}
\providecommand{\lcbrak}[1]{\ensuremath{\left\{#1\right.}}
\providecommand{\rcbrak}[1]{\ensuremath{\left.#1\right\}}}
\theoremstyle{remark}
\newtheorem{rem}{Remark}
\newcommand{\sgn}{\mathop{\mathrm{sgn}}}
\providecommand{\abs}[1]{\left\vert#1\right\vert}
\providecommand{\res}[1]{\Res\displaylimits_{#1}}
\providecommand{\norm}[1]{\left\lVert#1\right\rVert}
%\providecommand{\norm}[1]{\lVert#1\rVert}
\providecommand{\mtx}[1]{\mathbf{#1}}
\providecommand{\mean}[1]{E\left[ #1 \right]}
\providecommand{\fourier}{\overset{\mathcal{F}}{ \rightleftharpoons}}
%\providecommand{\hilbert}{\overset{\mathcal{H}}{ \rightleftharpoons}}
\providecommand{\system}{\overset{\mathcal{H}}{ \longleftrightarrow}}
%\newcommand{\solution}[2]{\textbf{Solution:}{#1}}
\newcommand{\solution}{\noindent \textbf{Solution: }}
\newcommand{\cosec}{\,\text{cosec}\,}
\providecommand{\dec}[2]{\ensuremath{\overset{#1}{\underset{#2}{\gtrless}}}}
\newcommand{\myvec}[1]{\ensuremath{\begin{pmatrix}#1\end{pmatrix}}}
\newcommand{\mydet}[1]{\ensuremath{\begin{vmatrix}#1\end{vmatrix}}}
\numberwithin{equation}{subsection}
\makeatletter
\@addtoreset{figure}{problem}
\makeatother
\let\StandardTheFigure\thefigure
\let\vec\mathbf
\renewcommand{\thefigure}{\theproblem}
\def\putbox#1#2#3{\makebox[0in][l]{\makebox[#1][l]{}\raisebox{\baselineskip}[0in][0in]{\raisebox{#2}[0in][0in]{#3}}}}
     \def\rightbox#1{\makebox[0in][r]{#1}}
     \def\centbox#1{\makebox[0in]{#1}}
     \def\topbox#1{\raisebox{-\baselineskip}[0in][0in]{#1}}
     \def\midbox#1{\raisebox{-0.5\baselineskip}[0in][0in]{#1}}
\vspace{3cm}
\title{Assignment 7}
\author{Priya Bhatia}
\maketitle
\newpage
%\tableofcontents
\bigskip
\renewcommand{\thefigure}{\theenumi}
\renewcommand{\thetable}{\theenumi}
\begin{abstract}
This document finds the coordinates of foot of perpendicular using Singular Value Decomposition
\end{abstract}
%
Download python codes from 
%
\begin{lstlisting}
https://github.com/priya6971/matrix_theory_EE5609/tree/master/Assignment7/codes
\end{lstlisting}
%
%
Download latex-tikz codes from 
%
\begin{lstlisting}
https://github.com/priya6971/matrix_theory_EE5609/tree/master/Assignment7
\end{lstlisting}
%
\section{\textbf{Problem}}
Determine the distance from the Z-axis to the plane 5x - 12y - 8 = 0
\section{\textbf{Solution}}
Equation of plane can be expressed as 
\begin{align}\label{eq1}
	\vec{n}^T\vec{x} &= c
\end{align}
Rewriting given equation of plane in \eqref{eq1} form
\begin{align}\label{eq2}
	\myvec{5 & -12 & 0}\myvec{x\\y\\z} &= 8
\end{align}
where the value of 
\begin{align}
    \vec{n} &= \myvec{5\\-12\\0} \\
    \vec{x} &= \myvec{x\\y\\z} \\
    c &= 8
\end{align}
We need to represent the equation of plane in parametric form,
\begin{equation}\label{eq3}
	\vec{Q} = \vec{p} + \lambda_1\vec{q} + \lambda_2\vec{r}
\end{equation}
Here $p$ is any point on plane and $\vec{q}, \vec{r}$ are two vectors parallel to plane and hence $\perp$ to $\vec{n}$. Now, we need to find these two vectors $\vec{q}$ and $\vec{r}$ which are $\perp$ to $\vec{n}$
\begin{align}
	\myvec{5 & -12 & 0}\myvec{a\\b\\c} = 0
	\implies 5a - 12b &= 0 \label{eq4}
\end{align}
Put $a=0$ and $c=1$ in \eqref{eq4}, $\implies b=0$\\
Put $a=1$ and $c=0$ in \eqref{eq4}, $\implies b=\frac{5}{12}$\\
Hence $\vec{q} = \myvec{1\\\frac{5}{12}\\0}, \vec{r} = \myvec{0\\0\\1}$\\
Let us find point $\vec{p}$ on the plane. Put $x=1,z=0$ in \eqref{eq2}, we get $\vec{p} = \myvec{1\\1\\0}$\\
Since given plane is parallel to Z-axis, we can use any point $P$ on Z-axis to compute shortest distance. 
\begin{equation}\label{eq5}
	\vec{P} = \myvec{0\\0\\0}
\end{equation}
Let $\vec{Q}$ be the point on plane with shortest distance to $\vec{P}$.
$\vec{Q}$ can be expressed in \eqref{eq4} form as
\begin{align}\label{eq6}
	\vec{Q} = \myvec{1\\1\\0} + \lambda_1\myvec{1\\\frac{5}{12}\\0} + \lambda_2\myvec{0\\0\\1}
\end{align}
Computation of Pseudo Inverse using SVD in order to determine the value of $\lambda_1$ and $\lambda_2$ :
\begin{align}
	\label{eq7}\myvec{1\\1\\0} + \lambda_1\myvec{1\\\frac{5}{12}\\0} + \lambda_2\myvec{0\\0\\1} &= \myvec{0\\0\\0}\\\label{eq8}
	\lambda_1\myvec{1\\\frac{5}{12}\\0} + \lambda_2\myvec{0\\0\\1} &= \myvec{-1\\-1\\0}\\\label{eq9}
	\myvec{1 & 0\\\frac{5}{12} & 0\\0 & 1} \myvec{\lambda_1 \\ \lambda_2} &=\myvec{-1\\-1\\0}\\\label{eq10}
	\vec{M}\vec{x} &= \vec{b}\\\label{eq11} 
	\implies\vec{x} &= \vec{M}^{+}\vec{b}
\end{align}
where,
\begin{align}
    \vec{M} &= \myvec{1 & 0\\\frac{5}{12} & 0\\0 & 1} \\
    \vec{x} &= \myvec{\lambda_1 \\ \lambda_2} \\
    \vec{b} &= \myvec{-1\\-1\\0}
\end{align}
Applying Singular Value Decomposition on $\vec{M}$,
\begin{align} \label{eq:eq_6}
    \vec{M}=\vec{U}\vec{S}\vec{V}^T
\end{align}
Where the columns of $\vec{V}$ are the eigenvectors of $\vec{M}^T\vec{M}$, the columns of $\vec{U}$ are the eigenvectors of $\vec{M}\vec{M}^T$ and $\vec{S}$ is diagonal matrix of Singular values of $\vec{M}^T\vec{M}$.
\begin{align}
    \vec{M}^T\vec{M} &= \myvec{\frac{169}{144} & 0\\0 & 1} \\
    \vec{M}\vec{M}^T &= \myvec{1 & \frac{5}{12} & 0\\\frac{5}{12} & \frac{25}{144} & 0\\ 0 & 0 & 1} 
\end{align}
As we know that,
\begin{align}
    \vec{U} \vec{S} \vec{V}^T \vec{x} = \vec{b} \nonumber \\
    \implies \vec{x} = \vec{V} \vec{S_+} \vec{U^T} \vec{b} \label{eq:eq_9}
\end{align}
Where $\vec{S_+}$ is Moore-Penrose Pseudo-Inverse of $\vec{S}$. Calculating eigenvalues of $\vec{M}\vec{M}^T$,
\begin{align}
    \mydet{\vec{M} \vec{M}^T - \lambda \vec{I}} = 0 \nonumber \\
    \implies \mydet{1-\lambda & \frac{5}{12} & 0 \\ \frac{5}{12} & \frac{25}{144}-\lambda & 0 \\ 0 & 0 & 1-\lambda} &= 0 \nonumber \\
    \implies \lambda^3 - \frac{313}{144}\lambda^2 + \frac{169}{144}\lambda =0 \nonumber
\end{align}
Hence eigenvalues of $\vec{M}\vec{M}^T$ are,
\begin{align} \label{eq:eq_10}
    \lambda_1 = \frac{169}{144}; \quad \lambda_2 = 1; \quad \lambda_3 =0
\end{align}
And the corresponding eigenvectors are,
\begin{align}
    \vec{u_1} = \myvec{1 \\ \frac{5}{12} \\ 0}; \quad \vec{u_2} = \myvec{0 \\ 0 \\ 1}; \quad
    \vec{u_3} = \myvec{-\frac{5}{12} \\ 1 \\ 0} \label{eq:eq_11} 
\end{align}
\begin{align} \label{eq:eq_13}
    \vec{U} = \myvec{1 & 0 & -\frac{5}{12} \\ \frac{5}{12} & 0 & 1 \\ 0 & 1 & 0}
\end{align}
Using values from \eqref{eq:eq_10},
\begin{align} \label{eq:eq_14}
    \vec{S} = \myvec{\frac{13}{12} & 0 \\ 0 & 1 \\ 0 & 0} 
\end{align}
Calculating the eigenvalues of $\vec{M}^T\vec{M}$,
\begin{align}
    \mydet{\vec{M}^T\vec{M} - \lambda \vec{I}} = 0 \nonumber \\
    \implies \mydet{\frac{169}{144}-\lambda & 0 \\ 0 & 1-\lambda} &= 0 \nonumber \\
    \implies \lambda^2 - \frac{313}{144}\lambda + \frac{169}{144} &= 0 \nonumber
\end{align}
Hence, eigenvalues of $\vec{M}^T\vec{M}$ are,
\begin{align}
    \lambda_4 = \frac{169}{144}; \quad \lambda_5 = 1 \nonumber
\end{align}
And the corresponding eigenvectors are,
\begin{align}
    \vec{v}_1 = \myvec{1 \\ 0}; \quad 
    \vec{v}_2 = \myvec{0 \\ 1} \label{eq:eq_15}
\end{align}
From \eqref{eq:eq_15} we obtain $\vec{V}$ as,
\begin{align} \label{eq:eq_16}
    \vec{V} = \myvec{1 & 0 \\ 0 & 1}
\end{align}
Now, we can compute $\textit{SVD}$ of $\vec{M}$ :
\begin{align}
	\label{eq16}\vec{M} &= \vec{U}\Vec{S}\vec{V}^T\\
	\label{eq17}\myvec{1 & 0\\\frac{5}{12} & 0 \\0 & 1} &= \myvec{1 & 0 & -\frac{5}{12}\\ \frac{5}{12} & 0 & 1 \\0 & 1 & 0}\myvec{\frac{13}{12} & 0 \\0 & 1\\0 & 0} \myvec{1 & 0\\0 & 1}\\
	\label{eq18}\vec{M}^{+} &= \vec{V}\Vec{S}^T\vec{U}^T\\
	\label{eq19}&= \myvec{1 & 0\\0 & 1}\myvec{\frac{13}{12} & 0 & 0\\0 & 1 & 0}\myvec{1 & \frac{5}{12} & 0\\ 0 & 0 & 1 \\-\frac{5}{12} & 1 & 0}\\
	\label{eq20}&=\myvec{\frac{144}{169} & \frac{60}{169} & 0\\0 & 0 & 1}
\end{align}
Substitute \eqref{eq20} in \eqref{eq11},
\begin{align}\label{23}
	\vec{x} &= \myvec{\frac{144}{169} & \frac{60}{169} & 0\\0 & 0 & 1}\myvec{-1\\-1\\0} \\
	\vec{x} &= \myvec{-\frac{204}{169}\\0} 
\end{align}
\begin{align}
	\implies\myvec{\lambda_1 \\ \lambda_2} &= \myvec{-\frac{204}{169}\\0}
\end{align}
Substituting $\lambda_1$, $\lambda_2$ in \eqref{eq6}
\begin{equation}
	\vec{Q} = \myvec{-\frac{204}{169}\\-\frac{85}{169}\\0}
\end{equation}
Distance between point $\vec{P}$ and $\vec{Q}$ is
\begin{align}
	\norm{\vec{P}-\vec{Q}} &= \sqrt{\left(-\frac{204}{169}\right)^2 +\left(-\frac{85}{169}\right)^2 + 0}\\
	\norm{\vec{P}-\vec{Q}} &= \frac{17}{13} 
\end{align}
Hence, the distance from the Z-axis to the plane $5x - 12y - 8 = 0$ is $\frac{17}{13}$. Now, we can verify the solution using Least Squares Method,
\begin{align}
	\vec{M}^T(\vec{b} - \vec{M}\vec{x}) &= 0\\
	\label{eq30}\implies \vec{M}^T\vec{M}\vec{x} &= \vec{M}^T\vec{b}
\end{align}
Substituting $\vec{M}, \vec{b}$ from \eqref{eq9} in \eqref{eq30}
\begin{align}
	\myvec{1 & 0\\\frac{5}{12} & 0\\0 & 1}\myvec{1 & \frac{5}{12} & 0\\0 & 0 & 1}\vec{x} &= \myvec{1 & \frac{5}{12} & 0\\0 & 0 & 1}\myvec{-1\\-1\\0}\\
	\myvec{\frac{169}{144} & 0\\0 & 1}\myvec{\lambda_1\\\lambda_2} &= \myvec{-\frac{17}{12}\\0}\\
	\implies\frac{169}{144}\lambda_1 &= -\frac{17}{12}\\
	\lambda_1 &= -\frac{17}{12} \times \frac{144}{169} = -\frac{204}{169}\\
	\text{and }\lambda_2 &= 0\\
	\label{31}\implies\vec{x} &= \myvec{-\frac{204}{169}\\0}
\end{align}
Comparing \eqref{23} and \eqref{31} solution is verified.
\end{document}