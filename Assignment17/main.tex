\documentclass[journal,12pt]{IEEEtran}
\usepackage{longtable}
\usepackage{setspace}
\usepackage{gensymb}
\singlespacing
\usepackage[cmex10]{amsmath}
\newcommand\myemptypage{
	\null
	\thispagestyle{empty}
	\addtocounter{page}{-1}
	\newpage
}
\usepackage{amsthm}
\usepackage{mdframed}
\usepackage{mathrsfs}
\usepackage{txfonts}
\usepackage{stfloats}
\usepackage{bm}
\usepackage{cite}
\usepackage{cases}
\usepackage{subfig}

\usepackage{longtable}
\usepackage{multirow}

\usepackage{enumitem}
\usepackage{mathtools}
\usepackage{steinmetz}
\usepackage{tikz}
\usepackage{circuitikz}
\usepackage{verbatim}
\usepackage{tfrupee}
\usepackage[breaklinks=true]{hyperref}
\usepackage{graphicx}
\usepackage{tkz-euclide}

\usetikzlibrary{calc,math}
\usepackage{listings}
    \usepackage{color}                                            %%
    \usepackage{array}                                            %%
    \usepackage{longtable}                                        %%
    \usepackage{calc}                                             %%
    \usepackage{multirow}                                         %%
    \usepackage{hhline}                                           %%
    \usepackage{ifthen}                                           %%
    \usepackage{lscape}     
\usepackage{multicol}
\usepackage{chngcntr}

\DeclareMathOperator*{\Res}{Res}

\renewcommand\thesection{\arabic{section}}
\renewcommand\thesubsection{\thesection.\arabic{subsection}}
\renewcommand\thesubsubsection{\thesubsection.\arabic{subsubsection}}

\renewcommand\thesectiondis{\arabic{section}}
\renewcommand\thesubsectiondis{\thesectiondis.\arabic{subsection}}
\renewcommand\thesubsubsectiondis{\thesubsectiondis.\arabic{subsubsection}}


\hyphenation{op-tical net-works semi-conduc-tor}
\def\inputGnumericTable{}                                 %%

\lstset{
%language=C,
frame=single, 
breaklines=true,
columns=fullflexible
}
\begin{document}
\onecolumn

\newtheorem{theorem}{Theorem}[section]
\newtheorem{problem}{Problem}
\newtheorem{proposition}{Proposition}[section]
\newtheorem{lemma}{Lemma}[section]
\newtheorem{corollary}[theorem]{Corollary}
\newtheorem{example}{Example}[section]
\newtheorem{definition}[problem]{Definition}

\newcommand{\BEQA}{\begin{eqnarray}}
\newcommand{\EEQA}{\end{eqnarray}}
\newcommand{\define}{\stackrel{\triangle}{=}}
\bibliographystyle{IEEEtran}
\raggedbottom
\setlength{\parindent}{0pt}
\providecommand{\mbf}{\mathbf}
\providecommand{\pr}[1]{\ensuremath{\Pr\left(#1\right)}}
\providecommand{\qfunc}[1]{\ensuremath{Q\left(#1\right)}}
\providecommand{\sbrak}[1]{\ensuremath{{}\left[#1\right]}}
\providecommand{\lsbrak}[1]{\ensuremath{{}\left[#1\right.}}
\providecommand{\rsbrak}[1]{\ensuremath{{}\left.#1\right]}}
\providecommand{\brak}[1]{\ensuremath{\left(#1\right)}}
\providecommand{\lbrak}[1]{\ensuremath{\left(#1\right.}}
\providecommand{\rbrak}[1]{\ensuremath{\left.#1\right)}}
\providecommand{\cbrak}[1]{\ensuremath{\left\{#1\right\}}}
\providecommand{\lcbrak}[1]{\ensuremath{\left\{#1\right.}}
\providecommand{\rcbrak}[1]{\ensuremath{\left.#1\right\}}}
\theoremstyle{remark}
\newtheorem{rem}{Remark}
\newcommand{\sgn}{\mathop{\mathrm{sgn}}}
\providecommand{\abs}[1]{\left\vert#1\right\vert}
\providecommand{\res}[1]{\Res\displaylimits_{#1}} 
\providecommand{\norm}[1]{\left\lVert#1\right\rVert}
%\providecommand{\norm}[1]{\lVert#1\rVert}
\providecommand{\mtx}[1]{\mathbf{#1}}
\providecommand{\mean}[1]{E\left[ #1 \right]}
\providecommand{\fourier}{\overset{\mathcal{F}}{ \rightleftharpoons}}
%\providecommand{\hilbert}{\overset{\mathcal{H}}{ \rightleftharpoons}}
\providecommand{\system}{\overset{\mathcal{H}}{ \longleftrightarrow}}
	%\newcommand{\solution}[2]{\textbf{Solution:}{#1}}
\newcommand{\solution}{\noindent \textbf{Solution: }}
\newcommand{\cosec}{\,\text{cosec}\,}
\providecommand{\dec}[2]{\ensuremath{\overset{#1}{\underset{#2}{\gtrless}}}}
\newcommand{\myvec}[1]{\ensuremath{\begin{pmatrix}#1\end{pmatrix}}}
\newcommand{\mydet}[1]{\ensuremath{\begin{vmatrix}#1\end{vmatrix}}}
\numberwithin{equation}{subsection}
\makeatletter
\@addtoreset{figure}{problem}
\makeatother
\let\StandardTheFigure\thefigure
\let\vec\mathbf
\renewcommand{\thefigure}{\theproblem}
\def\putbox#1#2#3{\makebox[0in][l]{\makebox[#1][l]{}\raisebox{\baselineskip}[0in][0in]{\raisebox{#2}[0in][0in]{#3}}}}
     \def\rightbox#1{\makebox[0in][r]{#1}}
     \def\centbox#1{\makebox[0in]{#1}}
     \def\topbox#1{\raisebox{-\baselineskip}[0in][0in]{#1}}
     \def\midbox#1{\raisebox{-0.5\baselineskip}[0in][0in]{#1}}
\vspace{3cm}
\title{Assignment 17}
\author{Mtech in AI Department\\Priya Bhatia\\AI20MTECH14015}
\maketitle
\bigskip
\renewcommand{\thefigure}{\theenumi}
\renewcommand{\thetable}{\theenumi}
\begin{abstract}
This document illustrates about the properties related to projection and its relevant proof.
\end{abstract}
%
Download the latex-tikz codes from 
%
\begin{lstlisting}
https://github.com/priya6971/matrix_theory_EE5609/tree/master/Assignment17
\end{lstlisting}
\section{\textbf{Problem}}
%
Let $E$ be a projection of $V$ and let $T$ be a linear operator on $V$. Prove that the range of $E$ is invariant under $T$ if and only if $ETE$ = $TE$. Prove that both the range and null space of $E$ are invariant under $T$ if and only if $ET$ = $TE$. 
%
\section{\textbf{Solution}}
\renewcommand{\thetable}{1}
\begin{longtable}{|c|l|}
    \hline
	\multirow{3}{*}{Proof of $ETE$ = $TE$} 
	& \\
	& Any projection $E$ is represented by a matrix that is a part of an identity matrix\\
	& Assume the Basis can be defined as follows:\\
	& $B$ = $\{\alpha_1,...,\alpha_r,....,\alpha_n\}$\\
	& such that $E_{ii}$ = 1 for i $\le$ r and 0 elsewhere\\
	& $E_{B}$ = $\myvec{$I$ & 0 \\ 0 & 0}$\\
	& where $I$ is the $r$ $\times$ $r$ matrix\\
	& Let $\alpha$ = $\brak{a_1,....,a_n}$\\
	& $\implies$ $T\brak{E_{\alpha}}$ = $T\brak{a_1,...,a_r,...0}$ = $\beta$ \\
	& If we assume $T$ to be invariant over the range $W$ of $E$, then $\beta$ $\in$ $W$ \\
	& $\beta$ = $\brak{\beta_1,...,\beta_r,...,0}$, $E_{\beta}$ = $\beta$\\
	& Therefore, $ETE$ = $TE$ \\
	&\\
	\hline
	\multirow{3}{*}{Proof of $ET$ = $TE$} 
	& \\
	& Consider the same assumption for basis $B$ and projection $E$ as defined above.\\
	& $B$ = $\{\alpha_1,...,\alpha_r,....,\alpha_n\}$\\
	& $E_{B}$ = $\myvec{$I$ & 0 \\ 0 & 0}$\\
	& where $I$ is the $r$ $\times$ $r$ matrix\\
	& Let $TE$ $\ne$ $ET$, then there exists some vector $V$ such that \\
	& $T\brak{a_1,...,a_r,...0}$ $\ne$ $\brak{Ta_1,...,Ta_r,...,0}$\\
	& But for this case, $T$ is not an invariant of $W$.\\
	& Assuming that $T$ is an invariant of $W$,$T\brak{a_1,...,a_r,...0}$ $\in$ $W$ for all $\alpha$ $\in$ $W$.\\
	& Therefore, $T\brak{a_1,...,a_r,...0}$ = $\brak{Ta_1,...,Ta_r,...,0}$ $\implies$ $ET$ = $TE$\\
	&\\
	\hline
	\multirow{3}{*}{Conclusion} & \\
	& Hence, it is proved that the range of $E$ is invariant under $T$ if and only if\\
	& $ETE$ = $TE$.\\
	& And both the range and null space of $E$ are invariant under $T$ if and only if\\
	& $TE$ = $ET$.\\
	&\\
	\hline
	\caption{Illustration of Proof}
    \label{table:1}
\end{longtable}
\end{document}