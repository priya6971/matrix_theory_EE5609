\documentclass[journal,12pt,twocolumn]{IEEEtran}

\usepackage{setspace}
\usepackage{gensymb}


\singlespacing

\usepackage[cmex10]{amsmath}
%\usepackage{amsthm}
%\interdisplaylinepenalty=2500
%\savesymbol{iint}
%\usepackage{txfonts}
%\restoresymbol{TXF}{iint}
%\usepackage{wasysym}
\usepackage{amsthm}

\usepackage{mathrsfs}
\usepackage{txfonts}
\usepackage{stfloats}
\usepackage{bm}
\usepackage{cite}
\usepackage{cases}
\usepackage{subfig}

\usepackage{longtable}
\usepackage{multirow}

\usepackage{enumitem}
\usepackage{mathtools}
\usepackage{steinmetz}
\usepackage{tikz}
\usepackage{circuitikz}
\usepackage{verbatim}
\usepackage{tfrupee}
\usepackage[breaklinks=true]{hyperref}

\usepackage{tkz-euclide} %loads TikZ and tkz-base

\usetikzlibrary{calc,math}
\usepackage{listings}
    \usepackage{color}                                          
    \usepackage{array}                                          
    \usepackage{longtable}                                      
    \usepackage{calc}                                          
    \usepackage{multirow}                                      
    \usepackage{hhline}                                        
    \usepackage{ifthen}
    \usepackage{lscape}    
\usepackage{multicol}
\usepackage{chngcntr}

\DeclareMathOperator*{\Res}{Res}

\renewcommand\thesection{\arabic{section}}
\renewcommand\thesubsection{\thesection.\arabic{subsection}}
\renewcommand\thesubsubsection{\thesubsection.\arabic{subsubsection}}

\renewcommand\thesectiondis{\arabic{section}}
\renewcommand\thesubsectiondis{\thesectiondis.\arabic{subsection}}
\renewcommand\thesubsubsectiondis{\thesubsectiondis.\arabic{subsubsection}}

\hyphenation{op-tical net-works semi-conduc-tor}
\def\inputGnumericTable{}                                 %%

\lstset{
%language=C,
frame=single,
breaklines=true,
columns=fullflexible
}

\begin{document}

\newtheorem{theorem}{Theorem}[section]
\newtheorem{problem}{Problem}
\newtheorem{proposition}{Proposition}[section]
\newtheorem{lemma}{Lemma}[section]
\newtheorem{corollary}[theorem]{Corollary}
\newtheorem{example}{Example}[section]
\newtheorem{definition}[problem]{Definition}

\newcommand{\BEQA}{\begin{eqnarray}}
\newcommand{\EEQA}{\end{eqnarray}}
\newcommand{\define}{\stackrel{\triangle}{=}}
\bibliographystyle{IEEEtran}
\providecommand{\mbf}{\mathbf}
\providecommand{\pr}[1]{\ensuremath{\Pr\left(#1\right)}}
\providecommand{\qfunc}[1]{\ensuremath{Q\left(#1\right)}}
\providecommand{\sbrak}[1]{\ensuremath{{}\left[#1\right]}}
\providecommand{\lsbrak}[1]{\ensuremath{{}\left[#1\right.}}
\providecommand{\rsbrak}[1]{\ensuremath{{}\left.#1\right]}}
\providecommand{\brak}[1]{\ensuremath{\left(#1\right)}}
\providecommand{\lbrak}[1]{\ensuremath{\left(#1\right.}}
\providecommand{\rbrak}[1]{\ensuremath{\left.#1\right)}}
\providecommand{\cbrak}[1]{\ensuremath{\left\{#1\right\}}}
\providecommand{\lcbrak}[1]{\ensuremath{\left\{#1\right.}}
\providecommand{\rcbrak}[1]{\ensuremath{\left.#1\right\}}}
\theoremstyle{remark}
\newtheorem{rem}{Remark}
\newcommand{\sgn}{\mathop{\mathrm{sgn}}}
\providecommand{\abs}[1]{\left\vert#1\right\vert}
\providecommand{\res}[1]{\Res\displaylimits_{#1}}
\providecommand{\norm}[1]{\left\lVert#1\right\rVert}
%\providecommand{\norm}[1]{\lVert#1\rVert}
\providecommand{\mtx}[1]{\mathbf{#1}}
\providecommand{\mean}[1]{E\left[ #1 \right]}
\providecommand{\fourier}{\overset{\mathcal{F}}{ \rightleftharpoons}}
%\providecommand{\hilbert}{\overset{\mathcal{H}}{ \rightleftharpoons}}
\providecommand{\system}{\overset{\mathcal{H}}{ \longleftrightarrow}}
%\newcommand{\solution}[2]{\textbf{Solution:}{#1}}
\newcommand{\solution}{\noindent \textbf{Solution: }}
\newcommand{\cosec}{\,\text{cosec}\,}
\providecommand{\dec}[2]{\ensuremath{\overset{#1}{\underset{#2}{\gtrless}}}}
\newcommand{\myvec}[1]{\ensuremath{\begin{pmatrix}#1\end{pmatrix}}}
\newcommand{\mydet}[1]{\ensuremath{\begin{vmatrix}#1\end{vmatrix}}}
\numberwithin{equation}{subsection}
\makeatletter
\@addtoreset{figure}{problem}
\makeatother
\let\StandardTheFigure\thefigure
\let\vec\mathbf
\renewcommand{\thefigure}{\theproblem}
\def\putbox#1#2#3{\makebox[0in][l]{\makebox[#1][l]{}\raisebox{\baselineskip}[0in][0in]{\raisebox{#2}[0in][0in]{#3}}}}
     \def\rightbox#1{\makebox[0in][r]{#1}}
     \def\centbox#1{\makebox[0in]{#1}}
     \def\topbox#1{\raisebox{-\baselineskip}[0in][0in]{#1}}
     \def\midbox#1{\raisebox{-0.5\baselineskip}[0in][0in]{#1}}
\vspace{3cm}
\title{Assignment 1}
\author{Priya Bhatia}
\maketitle
\newpage
%\tableofcontents
\bigskip
\renewcommand{\thefigure}{\theenumi}
\renewcommand{\thetable}{\theenumi}
\begin{abstract}
This document solves a problem from Lines and Planes,where we solve the given pair of linear equations.
\end{abstract}
Download all python codes from
\begin{lstlisting}
https://github.com/priya6971/matrix_theory_EE5609/tree/master/school/tree/master/training/design/codes
\end{lstlisting}
%
and latex-tikz codes from
%
\begin{lstlisting}
https://github.com/priya6971/matrix_theory_EE5609/tree/master/school/tree/master/training/design
\end{lstlisting}
%
\section{Problem}
Solve the following pair of linear equation
\begin{align}
\myvec{  158 & -378 \\ -378 & 152} x =
\myvec{  -74 \\ -604}
\end{align}
\section{Explanation}
Let the matrix is A and b is the vector.
So,Ax = b
Then we can calculate x = $A^{-1}$ . b
\begin{align}
\vec{A} = \myvec{158 & -378 \\ -378 & 152}
\end{align}
\begin{align}
\vec{b} = \myvec{-74 \\ -604}
\end{align}
\section{Solution}
\begin{align}
\vec{A} = \myvec{158 & -378 \\ -378 & 152}
\end{align}
\begin{align}
\vec{b} = \myvec{-74 \\ -604}
\end{align}
Augmented Matrix is : Ax = b
\begin{align}
\vec{R} = \myvec{  0 & -1 \\
1 & 0}
\end{align}
Now,multiply R on both sides, we get :
RAx = Rb
\begin{align}
\myvec{0 & -1 \\ 1 & 0}
\myvec{158 & -378 \\ -378 & 152} x = \nonumber \\
\myvec{0 & -1 \\ 1 & 0}
\myvec{-74 \\ -604}
\end{align}
After multiplication of matrix:
\begin{align}
\myvec{378 & -152 \\ -158 & -378} x =
\myvec{604 \\ -74}
\end{align}
Now, multiply the matrix A on both sides,we get:
ARAx = ARb
\begin{align}
\myvec{158 & -378 \\ -378 & 152}
\myvec{378 & -152 \\ -158 & -378} x = \nonumber \\
\myvec{158 & -378 \\ -378 & 152}
\myvec{604 \\ -74}
\end{align}
Now, after multiplication of above matrices:
\begin{align}
\myvec{0 & 118,868 \\ -118,868 & 0 } x =
\myvec{123404 \\ -239560 }
\end{align}
Now, both the rows is row/118,868:
\begin{align}
\myvec{0 & 1 \\ -1 & 0} x = \myvec{30851/29717 \\ -59890/29717}
\end{align}
As we know EE$^ {-1}$ = I,
Now inverse of the below matrix is:
\begin{align}
\myvec{0 & 1 \\-1 & 0} = \myvec{0 & -1 \\ 1 & 0}
\end{align}
Now multiply with the inverse of the Matrix in order to get the identity Matrix :
\begin{align}
\myvec{0 & -1 \\ 1 & 0}
\myvec{0 & 1 \\ -1 & 0} x = \nonumber \\ 
\myvec{0 & -1 \\ 1 & 0}
\myvec{30851/29717 \\ -59890/29717}
\end{align}
So, final matrix is after matrix multiplication:
\begin{align}
\myvec{1 & 0 \\ 0 & 1} x =
\myvec{59890/29717 \\ 30851/29717}
\end{align}
So, final result is :
\begin{align}
x = \myvec{59890/29717 & 30851/29717}
\end{align}

Approach 2 :

RARA x = RAR b where RARA = I, where I is the Identity Matrix

Now as we know Ax = b,

RARA x = RAR b
But as we know RARA = I,

Ix = RAR b
So, x = RAR b

Above expression illustrates that
A$^ {-1}$ = RAR because x = A$^ {-1}$b

So, by putting the values of R,A and b we can easily find out the value of x as follows:

RARA x = RAR b

Matrix Multiplication of RA is :
\begin{align}
\myvec{0 & -1 \\ -1 & 0}
\myvec{158 & -378 \\ -378 & -152} =
\myvec{378 & -152 \\ 158 & -378}
\end{align}
Now, using above resultant matrix RA we can evaluate 
RARA x = RAR b
\begin{align}
\myvec{378 & -152 \\ 158 & -378}
\myvec{378 & -152 \\ 158 & -378}x = \nonumber \\
\myvec{378 & -152 \\ 158 & -378}
\myvec{0 & -1 \\ -1 & 0}
\myvec{-74 \\ -604}
\end{align}
After doing matrix multiplication in LHS and RHS,
\begin{align}
\myvec{118868 & 0 \\ 0 & 118868}x =
\myvec{239560 \\ 123404}
\end{align}
Now, divide both the rows by 118868:
\begin{align}
\myvec{1 & 0 \\ 0 & 1}x =
\myvec{239560/118868 \\ 123404/118868}
\end{align}
After further calculations in the fractional result:
\begin{align}
x =\myvec{59890/29717 \\ 30851/29717}
\end{align}
\end{document}