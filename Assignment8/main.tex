\documentclass[journal,12pt,twocolumn]{IEEEtran}

\usepackage{setspace}
\usepackage{gensymb}


\singlespacing

\usepackage[cmex10]{amsmath}
%\usepackage{amsthm}
%\interdisplaylinepenalty=2500
%\savesymbol{iint}
%\usepackage{txfonts}
%\restoresymbol{TXF}{iint}
%\usepackage{wasysym}
\usepackage{amsthm}

\usepackage{mathrsfs}
\usepackage{txfonts}
\usepackage{stfloats}
\usepackage{bm}
\usepackage{cite}
\usepackage{cases}
\usepackage{subfig}

\usepackage{longtable}
\usepackage{multirow}

\usepackage{enumitem}
\usepackage{mathtools}
\usepackage{steinmetz}
\usepackage{tikz}
\usepackage{circuitikz}
\usepackage{verbatim}
\usepackage{tfrupee}
\usepackage[breaklinks=true]{hyperref}

\usepackage{tkz-euclide} %loads TikZ and tkz-base

\usetikzlibrary{calc,math}
\usepackage{listings}
    \usepackage{color}                                          
    \usepackage{array}                                          
    \usepackage{longtable}                                      
    \usepackage{calc}                                          
    \usepackage{multirow}                                      
    \usepackage{hhline}                                        
    \usepackage{ifthen}
    \usepackage{lscape}    
\usepackage{multicol}
\usepackage{chngcntr}

\DeclareMathOperator*{\Res}{Res}

\renewcommand\thesection{\arabic{section}}
\renewcommand\thesubsection{\thesection.\arabic{subsection}}
\renewcommand\thesubsubsection{\thesubsection.\arabic{subsubsection}}

\renewcommand\thesectiondis{\arabic{section}}
\renewcommand\thesubsectiondis{\thesectiondis.\arabic{subsection}}
\renewcommand\thesubsubsectiondis{\thesubsectiondis.\arabic{subsubsection}}

\hyphenation{op-tical net-works semi-conduc-tor}
\def\inputGnumericTable{}                                 %%

\lstset{
%language=C,
frame=single,
breaklines=true,
columns=fullflexible
}

\begin{document}

\newtheorem{theorem}{Theorem}[section]
\newtheorem{problem}{Problem}
\newtheorem{proposition}{Proposition}[section]
\newtheorem{lemma}{Lemma}[section]
\newtheorem{corollary}[theorem]{Corollary}
\newtheorem{example}{Example}[section]
\newtheorem{definition}[problem]{Definition}

\newcommand{\BEQA}{\begin{eqnarray}}
\newcommand{\EEQA}{\end{eqnarray}}
\newcommand{\define}{\stackrel{\triangle}{=}}
\bibliographystyle{IEEEtran}
\providecommand{\mbf}{\mathbf}
\providecommand{\pr}[1]{\ensuremath{\Pr\left(#1\right)}}
\providecommand{\qfunc}[1]{\ensuremath{Q\left(#1\right)}}
\providecommand{\sbrak}[1]{\ensuremath{{}\left[#1\right]}}
\providecommand{\lsbrak}[1]{\ensuremath{{}\left[#1\right.}}
\providecommand{\rsbrak}[1]{\ensuremath{{}\left.#1\right]}}
\providecommand{\brak}[1]{\ensuremath{\left(#1\right)}}
\providecommand{\lbrak}[1]{\ensuremath{\left(#1\right.}}
\providecommand{\rbrak}[1]{\ensuremath{\left.#1\right)}}
\providecommand{\cbrak}[1]{\ensuremath{\left\{#1\right\}}}
\providecommand{\lcbrak}[1]{\ensuremath{\left\{#1\right.}}
\providecommand{\rcbrak}[1]{\ensuremath{\left.#1\right\}}}
\theoremstyle{remark}
\newtheorem{rem}{Remark}
\newcommand{\sgn}{\mathop{\mathrm{sgn}}}
\providecommand{\abs}[1]{\left\vert#1\right\vert}
\providecommand{\res}[1]{\Res\displaylimits_{#1}}
\providecommand{\norm}[1]{\left\lVert#1\right\rVert}
%\providecommand{\norm}[1]{\lVert#1\rVert}
\providecommand{\mtx}[1]{\mathbf{#1}}
\providecommand{\mean}[1]{E\left[ #1 \right]}
\providecommand{\fourier}{\overset{\mathcal{F}}{ \rightleftharpoons}}
%\providecommand{\hilbert}{\overset{\mathcal{H}}{ \rightleftharpoons}}
\providecommand{\system}{\overset{\mathcal{H}}{ \longleftrightarrow}}
%\newcommand{\solution}[2]{\textbf{Solution:}{#1}}
\newcommand{\solution}{\noindent \textbf{Solution: }}
\newcommand{\cosec}{\,\text{cosec}\,}
\providecommand{\dec}[2]{\ensuremath{\overset{#1}{\underset{#2}{\gtrless}}}}
\newcommand{\myvec}[1]{\ensuremath{\begin{pmatrix}#1\end{pmatrix}}}
\newcommand{\mydet}[1]{\ensuremath{\begin{vmatrix}#1\end{vmatrix}}}
\numberwithin{equation}{subsection}
\makeatletter
\@addtoreset{figure}{problem}
\makeatother
\let\StandardTheFigure\thefigure
\let\vec\mathbf
\renewcommand{\thefigure}{\theproblem}
\def\putbox#1#2#3{\makebox[0in][l]{\makebox[#1][l]{}\raisebox{\baselineskip}[0in][0in]{\raisebox{#2}[0in][0in]{#3}}}}
     \def\rightbox#1{\makebox[0in][r]{#1}}
     \def\centbox#1{\makebox[0in]{#1}}
     \def\topbox#1{\raisebox{-\baselineskip}[0in][0in]{#1}}
     \def\midbox#1{\raisebox{-0.5\baselineskip}[0in][0in]{#1}}
\vspace{3cm}
\title{Assignment 8}
\author{Priya Bhatia}
\maketitle
\newpage
%\tableofcontents
\bigskip
\renewcommand{\thefigure}{\theenumi}
\renewcommand{\thetable}{\theenumi}
\begin{abstract}
This document finds the solutions of given matrix by row reduction.
\end{abstract}
%
Download python codes from 
%
\begin{lstlisting}
https://github.com/priya6971/matrix_theory_EE5609/tree/master/Assignment8/codes
\end{lstlisting}
%
%
%
Download latex-tikz codes from 
%
\begin{lstlisting}
https://github.com/priya6971/matrix_theory_EE5609/tree/master/Assignment8
\end{lstlisting}
%
\section{\textbf{Problem}}
If 
\begin{align}
    A &= \myvec{3 & -1 & 2 \\
                2 & 1 & 1 \\
                1 & -3 & 0}
\end{align}     
Find all solutions of $AX = 0$ by row reducing $A$.
\section{\textbf{Solution}}
For the given equation $AX = 0$ can be defined as follows:
\begin{align}\label{eq_1}
    \myvec{3 & -1 & 2 \\
                2 & 1 & 1 \\
                1 & -3 & 0}
    \myvec{x_1 \\ x_2 \\ x_3} &=
    \myvec{0 \\ 0 \\ 0}
\end{align}
Now, we can apply Row Reduction Methodology of matrix $A$ :
\begin{align}
\myvec{3&-1&2&0\\2&1&1&0\\1&-3&0&0} 
&\xleftrightarrow{R_1 
= R_1+R_2}\myvec{5&0&3&0\\2&1&1&0\\1&-3&0&0}\\
&\xleftrightarrow{R_2 = R_2-2R_3}\myvec{5&0&3&0\\0&7&1&0\\1&-3&0&0}\\
&\xleftrightarrow{R_3 = 
R_3 -\frac{1}{5}R_1}\myvec{5&0&3&0\\0&7&1&0\\0&-3&-\frac{3}{5}&0}\\
&\xleftrightarrow{R_1 =  \frac{1}{5}R_1}\myvec{1&0&\frac{3}{5}&0\\0&7&1&0\\0&-3&-\frac{3}{5}&0}\\
&\xleftrightarrow{R_2 = \frac{1}{7}R_2}\myvec{1&0&\frac{3}{5}&0\\0&1&\frac{1}{7}&0\\0&-3&-\frac{3}{5}&0}\\
&\xleftrightarrow{R_3 = R_3 + 3R_2}\myvec{1&0&\frac{3}{5}&0\\0&1&\frac{1}{7}&0\\0&0&-\frac{6}{35}&0}\\
&\xleftrightarrow{R_3 = -\frac{35}{6} R_3}\myvec{1&0&\frac{3}{5}&0\\0&1&\frac{1}{7}&0\\0&0&1&0} \\
&\xleftrightarrow{R_2 = R_2 - \frac{1}{7} R_3}\myvec{1&0&\frac{3}{5}&0\\0&1&0&0\\0&0&1&0} \\
&\xleftrightarrow{R_1 = R_1 - \frac{3}{5} R_3}\myvec{1&0&0&0\\0&1&0&0\\0&0&1&0} 
\end{align} 
So, as we can see the only solution we got after row reducing of matrix $A$ is zero vector.
Thus, the solution is :
\begin{align}
    \myvec{x_1 \\ x_2 \\ x_3} = \myvec{0 \\ 0 \\ 0}
\end{align}
\end{document}