\documentclass[journal,12pt]{IEEEtran}
\usepackage{longtable}
\usepackage{setspace}
\usepackage{gensymb}
\singlespacing
\usepackage[cmex10]{amsmath}
\newcommand\myemptypage{
	\null
	\thispagestyle{empty}
	\addtocounter{page}{-1}
	\newpage
}
\usepackage{amsthm}
\usepackage{mdframed}
\usepackage{mathrsfs}
\usepackage{txfonts}
\usepackage{stfloats}
\usepackage{bm}
\usepackage{cite}
\usepackage{cases}
\usepackage{subfig}

\usepackage{longtable}
\usepackage{multirow}

\usepackage{enumitem}
\usepackage{mathtools}
\usepackage{steinmetz}
\usepackage{tikz}
\usepackage{circuitikz}
\usepackage{verbatim}
\usepackage{tfrupee}
\usepackage[breaklinks=true]{hyperref}
\usepackage{graphicx}
\usepackage{tkz-euclide}

\usetikzlibrary{calc,math}
\usepackage{listings}
    \usepackage{color}                                            %%
    \usepackage{array}                                            %%
    \usepackage{longtable}                                        %%
    \usepackage{calc}                                             %%
    \usepackage{multirow}                                         %%
    \usepackage{hhline}                                           %%
    \usepackage{ifthen}                                           %%
    \usepackage{lscape}     
\usepackage{multicol}
\usepackage{chngcntr}

\DeclareMathOperator*{\Res}{Res}

\renewcommand\thesection{\arabic{section}}
\renewcommand\thesubsection{\thesection.\arabic{subsection}}
\renewcommand\thesubsubsection{\thesubsection.\arabic{subsubsection}}

\renewcommand\thesectiondis{\arabic{section}}
\renewcommand\thesubsectiondis{\thesectiondis.\arabic{subsection}}
\renewcommand\thesubsubsectiondis{\thesubsectiondis.\arabic{subsubsection}}


\hyphenation{op-tical net-works semi-conduc-tor}
\def\inputGnumericTable{}                                 %%

\lstset{
%language=C,
frame=single, 
breaklines=true,
columns=fullflexible
}
\begin{document}
\onecolumn

\newtheorem{theorem}{Theorem}[section]
\newtheorem{problem}{Problem}
\newtheorem{proposition}{Proposition}[section]
\newtheorem{lemma}{Lemma}[section]
\newtheorem{corollary}[theorem]{Corollary}
\newtheorem{example}{Example}[section]
\newtheorem{definition}[problem]{Definition}

\newcommand{\BEQA}{\begin{eqnarray}}
\newcommand{\EEQA}{\end{eqnarray}}
\newcommand{\define}{\stackrel{\triangle}{=}}
\bibliographystyle{IEEEtran}
\raggedbottom
\setlength{\parindent}{0pt}
\providecommand{\mbf}{\mathbf}
\providecommand{\pr}[1]{\ensuremath{\Pr\left(#1\right)}}
\providecommand{\qfunc}[1]{\ensuremath{Q\left(#1\right)}}
\providecommand{\sbrak}[1]{\ensuremath{{}\left[#1\right]}}
\providecommand{\lsbrak}[1]{\ensuremath{{}\left[#1\right.}}
\providecommand{\rsbrak}[1]{\ensuremath{{}\left.#1\right]}}
\providecommand{\brak}[1]{\ensuremath{\left(#1\right)}}
\providecommand{\lbrak}[1]{\ensuremath{\left(#1\right.}}
\providecommand{\rbrak}[1]{\ensuremath{\left.#1\right)}}
\providecommand{\cbrak}[1]{\ensuremath{\left\{#1\right\}}}
\providecommand{\lcbrak}[1]{\ensuremath{\left\{#1\right.}}
\providecommand{\rcbrak}[1]{\ensuremath{\left.#1\right\}}}
\theoremstyle{remark}
\newtheorem{rem}{Remark}
\newcommand{\sgn}{\mathop{\mathrm{sgn}}}
\providecommand{\abs}[1]{\left\vert#1\right\vert}
\providecommand{\res}[1]{\Res\displaylimits_{#1}} 
\providecommand{\norm}[1]{\left\lVert#1\right\rVert}
%\providecommand{\norm}[1]{\lVert#1\rVert}
\providecommand{\mtx}[1]{\mathbf{#1}}
\providecommand{\mean}[1]{E\left[ #1 \right]}
\providecommand{\fourier}{\overset{\mathcal{F}}{ \rightleftharpoons}}
%\providecommand{\hilbert}{\overset{\mathcal{H}}{ \rightleftharpoons}}
\providecommand{\system}{\overset{\mathcal{H}}{ \longleftrightarrow}}
	%\newcommand{\solution}[2]{\textbf{Solution:}{#1}}
\newcommand{\solution}{\noindent \textbf{Solution: }}
\newcommand{\cosec}{\,\text{cosec}\,}
\providecommand{\dec}[2]{\ensuremath{\overset{#1}{\underset{#2}{\gtrless}}}}
\newcommand{\myvec}[1]{\ensuremath{\begin{pmatrix}#1\end{pmatrix}}}
\newcommand{\mydet}[1]{\ensuremath{\begin{vmatrix}#1\end{vmatrix}}}
\numberwithin{equation}{subsection}
\makeatletter
\@addtoreset{figure}{problem}
\makeatother
\let\StandardTheFigure\thefigure
\let\vec\mathbf
\renewcommand{\thefigure}{\theproblem}
\def\putbox#1#2#3{\makebox[0in][l]{\makebox[#1][l]{}\raisebox{\baselineskip}[0in][0in]{\raisebox{#2}[0in][0in]{#3}}}}
     \def\rightbox#1{\makebox[0in][r]{#1}}
     \def\centbox#1{\makebox[0in]{#1}}
     \def\topbox#1{\raisebox{-\baselineskip}[0in][0in]{#1}}
     \def\midbox#1{\raisebox{-0.5\baselineskip}[0in][0in]{#1}}
\vspace{3cm}
\title{Assignment 19}
\author{Mtech in AI Department\\Priya Bhatia\\AI20MTECH14015}
\maketitle
\bigskip
\renewcommand{\thefigure}{\theenumi}
\renewcommand{\thetable}{\theenumi}
\begin{abstract}
This document illustrates the concept of orthonormal basis.
\end{abstract}
%
Download the latex-tikz codes from 
%
\begin{lstlisting}
https://github.com/priya6971/matrix_theory_EE5609/tree/master/Assignment19
\end{lstlisting}
\section{\textbf{Problem}}
%
Let $\{u_1,u_2,...,u_n\}$ be an orthonormal basis of $C^n$ as column vectors. Let $M$ = $\{u_1,u_2,...,u_k\}$ and $N$ = $\{u_{k+1},u_{k+2},...,u_n\}$ and $P$ be the diagonal $k \times k$ matrix with diagonal entries $\alpha_1,\alpha_2,....,\alpha_k$ $\in$ $R$. Then which of the following is true?

1. Rank($\vec{MP{M}^*}$) = $\vec{k}$ whenever $\alpha_i$ $\ne$ $\alpha_j$, 1 $\leq$ $i$, $j$ $\leq$ $k$

2. Trace($\vec{MP{M}^*}$) = $\sum_{i=1}^{k}\alpha_i$

3. Rank($\vec{{M}^*N}$) = min($k$,$n-k$)

4. Rank($\vec{M{M}^*}+\vec{N{N}^*}$) $<$ $n$
%
\section{\textbf{Definitions}}
\renewcommand{\thetable}{1}
\begin{table}[ht!]
\centering
\begin{tabular}{|c|l|}
    \hline
	\multirow{3}{*}{Orthonormal Basis} 
	& \\
	& $B$ = $\{u_1,u_2,...,u_n\}$ is the Orthonormal basis for $C^n$ if it generates every vector $C^n$\\
	& and the inner product $<u_i,u_j>$ = $0$ if $i$ $\ne$ $j$.\\
	& That is the vectors are mutually perpendicular\\
	& and $<u_i,u_j>$ = $1$ otherwise. \\
	&\\
	\hline
	\multirow{3}{*}{Trace} 
	&\\
	& Trace of a square matrix $A$, denoted by $\Vec{tr(A)}$ is defined to be the sum of elements\\
	& on the main diagonal(from the upper left to lower right) of $A$\\
	& Some useful properties of Trace : \\
	&  $\Vec{tr(AB)}$ =  $\Vec{tr(BA)}$, where $A$ is the $m$ $\times$ $n$ matrix and $B$ is the $n$ $\times$ $m$ matrix\\  
	&\\
	\hline
	\multirow{3}{*}{Basis Theorem} 
	&\\
	& A nonempty subset of nonzero vectors in $R^n$ is called an orthogonal set\\
	& if every pair of distinct vectors in the set is orthogonal. Any Orthogonal sets\\
	&  of vectors are automatically linearly independent and if $A$ matrix columns are\\
	& linearly independent,then it is invertible.\\
	&\\
    \hline
\end{tabular}
\label{table:1}
    \caption{Definitions}
\end{table}
\newpage
\section{\textbf{Solution}}
\renewcommand{\thetable}{2}
\begin{longtable}{|c|l|}
    \hline
	\multirow{3}{*}{Rank($\vec{MP{M}^*}$) = $\vec{k}$} 
	& \\
	& Consider orthogonal vectors,\\
	& $\vec{u_1}$ = \myvec{1\\0\\0\\0}; $\vec{u_2}$ = \myvec{0\\1\\0\\0}\\
	& $\vec{u_3}$ = \myvec{0\\0\\1\\0}; $\vec{u_4}$ = \myvec{0\\0\\0\\1}\\
	& Consider $\vec{k}$ = 2, then \\
	& $\vec{M}$ = $\myvec{u_1&u_2}$ = $\myvec{1&0\\0&1\\0&0\\0&0}$\\
	& $\vec{M^*}$ = $\myvec{1&0&0&0\\0&1&0&0}$\\
	& $\vec{P}$ = $\myvec{\alpha_1&0\\0&\alpha_2}$\\
	& $\vec{MPM^{*}}$ = $\myvec{\alpha_1&0&0&0\\
	                       0&\alpha_2&0&0\\
	                       0&0&0&0\\
	                       0&0&0&0}$\\
	& $\implies$ Rank($\vec{MPM^{*}}$) $\le$ 2 (which is the value of $k$)\\
	& (It depends on diagonal values $\alpha_1$ and $\alpha_2$)\\
	& Rank($\vec{MPM^{*}}$) is not always $k$. \\
	& It can be less than k if any of the entries in $\alpha_1,\alpha_2,....,\alpha_k$ is 0.\\
	& Thus, Rank($\vec{MP{M}^*}$) $\ne$ $\vec{k}$\\
	& Thus, the given statement is false\\
	&\\
	\hline
	\multirow{3}{*}{Trace($\vec{MP{M}^*}$) = $\sum_{i=1}^{k}\alpha_i$} & \\
	& Consider $\vec{MP}$ = $\vec{A}$ and $\vec{M^{*}}$ = $\vec{B}$\\
	& Using Properties, Trace$\vec{\brak{AB}}$ = Trace$\vec{\brak{BA}}$\\
	& We can say, Trace($\vec{MP{M}^*}$) = Trace($\vec{{M}^*MP}$)\\
	& $\vec{M}$ = $\myvec{u_1&u_2&u_3&....&u_k}$ \\
	& $\vec{M^*}$ = $\myvec{\Bar{u_1}\\\Bar{u_2}\\\Bar{u_3}\\.\\.\\.\\\Bar{u_k}}$ \\
	&\\
	& $\vec{M^{*}M}$ = $\myvec{\Bar{u_1}u_1&0&0&...&0\\
	                           0&\Bar{u_2}u_2&0&...&0\\
	                           0&0&\Bar{u_3}u_3&...&0\\
	                           .&.&.&...&.\\
	                           0&0&0&...&\Bar{u_k}u_k}$\\
	 & (Refer to Properties mentioned in Orthonormal Basis in Definition section\\
	 & that is $<u_i,u_j>$ = $0$ if $i$ $\ne$ $j$)\\
	 & \\
	 & $\vec{M^{*}M}$ = $\myvec{1&0&0&...&0\\
	                           0&1&0&...&0\\
	                           0&0&1&...&0\\
	                           .&.&.&...&.\\
	                           0&0&0&...&1}$\\
	& (Refer to Properties mentioned in Orthonormal Basis in Definition section\\
	& that is $<u_i,u_j>$ = $1$ if $i$ = $j$)\\
    & $\vec{M^*M}$ = $\vec{I^{k}}$\\
    & $\vec{M^*MP}$ = $\vec{I^{k}P}$ = $\vec{P}$\\
    & Trace($\vec{{M}^*MP}$) = Trace($\vec{I^{k}P}$) =  Trace($\vec{P}$) = $\sum_{i=1}^{k}\alpha_i$\\
    & (Refer Definition section of Trace, it is sum of elements on the main diagonal)\\
    & So, the given statement is true \\
	& \\
	\hline
	\multirow{3}{*}{Rank($\vec{{M}^*N}$) = min($k$,$n-k$)} 
	& \\
	& $\vec{M}$ = $\{u_1,u_2,...,u_k\}$ and $\vec{N}$ = $\{u_{k+1},u_{k+2},...,u_n\}$ \\
	& Consider orthogonal vectors,\\
	& $\vec{u_1}$ = \myvec{1\\0\\0\\0}; $\vec{u_2}$ = \myvec{0\\1\\0\\0}\\
	& $\vec{u_3}$ = \myvec{0\\0\\1\\0}; $\vec{u_4}$ = \myvec{0\\0\\0\\1}\\
	& Consider $k$ = 2, then \\
	& $\vec{M}$ = $\myvec{u_1&u_2}$ = $\myvec{1&0\\0&1\\0&0\\0&0}$\\
	& $\vec{M^*}$ = $\myvec{1&0&0&0\\0&1&0&0}$\\
	& $\vec{N}$ = $\myvec{u_3&u_4}$ = $\myvec{0&0\\0&0\\1&0\\0&1}$\\
	& $\vec{M^*N}$ = $\myvec{0&0\\0&0}$\\
	& Rank($\vec{M^*N}$) = 0\\
	& But, min($k$,$n-k$) = \brak{2,2} = 2 \\
	& And, this is clear from above that Rank($\vec{{M}^*N}$) $\ne$ min($k$,$n-k$)\\
	& Thus, above statement is false \\
	&\\
	\hline
	\multirow{3}{*}{Rank($\vec{M{M}^*}+\vec{N{N}^*}$) $<$ $n$} 
	& \\
	& Rank($\vec{M}$) = Rank($\vec{M^*}$)\\
	& Rank($\vec{N}$) = Rank($\vec{N^*}$)\\
	& Rank($\vec{M}$+$\vec{N}$) $\le$ Rank($\vec{M}$) + Rank($\vec{N}$)\\
	& $\vec{M}$ = $\{u_1,u_2,...,u_k\}$ and $\vec{N}$ = $\{u_{k+1},u_{k+2},...,u_n\}$ \\
	& Consider orthogonal vectors,\\
	& $\vec{u_1}$ = \myvec{1\\0\\0\\0}; $\vec{u_2}$ = \myvec{0\\1\\0\\0}\\
	& $\vec{u_3}$ = \myvec{0\\0\\1\\0}; $\vec{u_4}$ = \myvec{0\\0\\0\\1}\\
	& Consider $k$ = 2, then \\
	& $\vec{M}$ = $\myvec{u_1&u_2}$ = $\myvec{1&0\\0&1\\0&0\\0&0}$\\
	& Rank($\vec{M}$) = $2$ = $k$\\
	& $\vec{N}$ = $\myvec{u_3&u_4}$ = $\myvec{0&0\\0&0\\1&0\\0&1}$\\
	& Rank($\vec{N}$) = $2$ = $n-k$\\
	& Thus, Rank($\vec{M{M}^*}+\vec{N{N}^*}$) = Rank($\vec{M}+\vec{N}$) = 4 = $n$\\
	& Thus, above statement is false \\
	&\\
	\hline
	\caption{Finding of True and False Statements}
    \label{table:2}
\end{longtable}
\section{\textbf{Conclusion}}
\renewcommand{\thetable}{3}
\begin{longtable}{|c|l|}
    \hline
	\multirow{3}{*}{Rank($\vec{MP{M}^*}$) = $\vec{k}$} 
	& \\
	& False \\
	&\\
	\hline
	\multirow{3}{*}{{Trace($\vec{MP{M}^*}$) = $\sum_{i=1}^{k}\alpha_i$}} 
	& \\
	& True \\
	& \\
	\hline
	\multirow{3}{*}{Rank($\vec{{M}^*N}$) = min($k$,$n-k$)} 
	& \\
	& False \\
	& \\
	\hline
	\multirow{3}{*}{Rank($\vec{M{M}^*}+\vec{N{N}^*}$) $<$ $n$} 
	& \\
	& False \\
	& \\
	\hline
	\caption{Conclusion of above Solutions}
    \label{table:3}
\end{longtable}
\end{document}