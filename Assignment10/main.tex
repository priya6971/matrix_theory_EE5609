\documentclass[journal,12pt,twocolumn]{IEEEtran}

\usepackage{setspace}
\usepackage{gensymb}


\singlespacing

\usepackage[cmex10]{amsmath}
%\usepackage{amsthm}
%\interdisplaylinepenalty=2500
%\savesymbol{iint}
%\usepackage{txfonts}
%\restoresymbol{TXF}{iint}
%\usepackage{wasysym}
\usepackage{amsthm}

\usepackage{mathrsfs}
\usepackage{txfonts}
\usepackage{stfloats}
\usepackage{bm}
\usepackage{cite}
\usepackage{cases}
\usepackage{subfig}

\usepackage{longtable}
\usepackage{multirow}

\usepackage{enumitem}
\usepackage{mathtools}
\usepackage{steinmetz}
\usepackage{tikz}
\usepackage{circuitikz}
\usepackage{verbatim}
\usepackage{tfrupee}
\usepackage[breaklinks=true]{hyperref}

\usepackage{tkz-euclide} %loads TikZ and tkz-base

\usetikzlibrary{calc,math}
\usepackage{listings}
    \usepackage{color}                                          
    \usepackage{array}                                          
    \usepackage{longtable}                                      
    \usepackage{calc}                                          
    \usepackage{multirow}                                      
    \usepackage{hhline}                                        
    \usepackage{ifthen}
    \usepackage{lscape}    
\usepackage{multicol}
\usepackage{chngcntr}

\DeclareMathOperator*{\Res}{Res}

\renewcommand\thesection{\arabic{section}}
\renewcommand\thesubsection{\thesection.\arabic{subsection}}
\renewcommand\thesubsubsection{\thesubsection.\arabic{subsubsection}}

\renewcommand\thesectiondis{\arabic{section}}
\renewcommand\thesubsectiondis{\thesectiondis.\arabic{subsection}}
\renewcommand\thesubsubsectiondis{\thesubsectiondis.\arabic{subsubsection}}

\hyphenation{op-tical net-works semi-conduc-tor}
\def\inputGnumericTable{}                                 %%

\lstset{
%language=C,
frame=single,
breaklines=true,
columns=fullflexible
}

\begin{document}

\newtheorem{theorem}{Theorem}[section]
\newtheorem{problem}{Problem}
\newtheorem{proposition}{Proposition}[section]
\newtheorem{lemma}{Lemma}[section]
\newtheorem{corollary}[theorem]{Corollary}
\newtheorem{example}{Example}[section]
\newtheorem{definition}[problem]{Definition}

\newcommand{\BEQA}{\begin{eqnarray}}
\newcommand{\EEQA}{\end{eqnarray}}
\newcommand{\define}{\stackrel{\triangle}{=}}
\bibliographystyle{IEEEtran}
\providecommand{\mbf}{\mathbf}
\providecommand{\pr}[1]{\ensuremath{\Pr\left(#1\right)}}
\providecommand{\qfunc}[1]{\ensuremath{Q\left(#1\right)}}
\providecommand{\sbrak}[1]{\ensuremath{{}\left[#1\right]}}
\providecommand{\lsbrak}[1]{\ensuremath{{}\left[#1\right.}}
\providecommand{\rsbrak}[1]{\ensuremath{{}\left.#1\right]}}
\providecommand{\brak}[1]{\ensuremath{\left(#1\right)}}
\providecommand{\lbrak}[1]{\ensuremath{\left(#1\right.}}
\providecommand{\rbrak}[1]{\ensuremath{\left.#1\right)}}
\providecommand{\cbrak}[1]{\ensuremath{\left\{#1\right\}}}
\providecommand{\lcbrak}[1]{\ensuremath{\left\{#1\right.}}
\providecommand{\rcbrak}[1]{\ensuremath{\left.#1\right\}}}
\theoremstyle{remark}
\newtheorem{rem}{Remark}
\newcommand{\sgn}{\mathop{\mathrm{sgn}}}
\providecommand{\abs}[1]{\left\vert#1\right\vert}
\providecommand{\res}[1]{\Res\displaylimits_{#1}}
\providecommand{\norm}[1]{\left\lVert#1\right\rVert}
%\providecommand{\norm}[1]{\lVert#1\rVert}
\providecommand{\mtx}[1]{\mathbf{#1}}
\providecommand{\mean}[1]{E\left[ #1 \right]}
\providecommand{\fourier}{\overset{\mathcal{F}}{ \rightleftharpoons}}
%\providecommand{\hilbert}{\overset{\mathcal{H}}{ \rightleftharpoons}}
\providecommand{\system}{\overset{\mathcal{H}}{ \longleftrightarrow}}
%\newcommand{\solution}[2]{\textbf{Solution:}{#1}}
\newcommand{\solution}{\noindent \textbf{Solution: }}
\newcommand{\cosec}{\,\text{cosec}\,}
\providecommand{\dec}[2]{\ensuremath{\overset{#1}{\underset{#2}{\gtrless}}}}
\newcommand{\myvec}[1]{\ensuremath{\begin{pmatrix}#1\end{pmatrix}}}
\newcommand{\mydet}[1]{\ensuremath{\begin{vmatrix}#1\end{vmatrix}}}
\numberwithin{equation}{subsection}
\makeatletter
\@addtoreset{figure}{problem}
\makeatother
\let\StandardTheFigure\thefigure
\let\vec\mathbf
\renewcommand{\thefigure}{\theproblem}
\def\putbox#1#2#3{\makebox[0in][l]{\makebox[#1][l]{}\raisebox{\baselineskip}[0in][0in]{\raisebox{#2}[0in][0in]{#3}}}}
     \def\rightbox#1{\makebox[0in][r]{#1}}
     \def\centbox#1{\makebox[0in]{#1}}
     \def\topbox#1{\raisebox{-\baselineskip}[0in][0in]{#1}}
     \def\midbox#1{\raisebox{-0.5\baselineskip}[0in][0in]{#1}}
\vspace{3cm}
\title{Assignment 10}
\author{Priya Bhatia}
\maketitle
\newpage
%\tableofcontents
\bigskip
\renewcommand{\thefigure}{\theenumi}
\renewcommand{\thetable}{\theenumi}
\begin{abstract}
This document gives an explicit description of the vectors in $R^5$ which are linear combination of the vectors.
\end{abstract}
%
Download latex-tikz codes from 
%
\begin{lstlisting}
https://github.com/priya6971/matrix_theory_EE5609/tree/master/Assignment10
\end{lstlisting}
%
%
Download python codes from 
%
\begin{lstlisting}
https://github.com/priya6971/matrix_theory_EE5609/tree/master/Assignment10/codes
\end{lstlisting}
%
\section{\textbf{Problem}}
Give an explicit description of the type $b_j=\sum_{i=1}^{r} b_{ki}R_{ij}$ for the vectors
\begin{align}
    \beta &= (b_1,b_2,b_3,b_4,b_5) \nonumber
\end{align}
in $R^5$ which are linear combinations of the vectors
\begin{align}
    \alpha_1 &= (1,0,2,1,-1), \\
    \alpha_2 &= (-1,2,-4,2,0), \\
    \alpha_3 &= (2,-1,5,2,1), \\
    \alpha_4 &= (2,1,3,5,2)
\end{align}
\section{\textbf{Solution}}
Above matrix represented as: $Ax = \beta$
\begin{align}
    \myvec{1 & -1 & 2 & 2 \\
                    0 & 2 & -1 & 1 \\
                    2 & -4 & 5 & 3 \\
                    1 & 2 & 2 & 5 \\
                    -1 & 0 & 1 & 2}x &= \beta\\
    \myvec{1 & -1 & 2 & 2 \\
                    0 & 2 & -1 & 1 \\
                    2 & -4 & 5 & 3 \\
                    1 & 2 & 2 & 5 \\
                    -1 & 0 & 1 & 2}x &= 
                    \myvec{b_1 \\ b_2 \\ b_3\\ b_4 \\ b_5} \label{eq_2}
\end{align}
$AX^T = Y^T$, where $A$, $X$ and $Y$ matrices have shown below:
\begin{align}
    A &= \myvec{1 & -1 & 2 & 2 \\
                    0 & 2 & -1 & 1 \\
                    2 & -4 & 5 & 3 \\
                    1 & 2 & 2 & 5 \\
                    -1 & 0 & 1 & 2} \\
    X &= \myvec{\frac{67}{100}&-\frac{167}{100}&-2&\frac{133}{100}\\
           \frac{1}{2}&-\frac{3}{2}&-\frac{5}{2}&\frac{3}{2}\\
           -\frac{4}{25}&\frac{117}{100}&\frac{3}{2}&-\frac{83}{100}\\
           -\frac{1}{2}&-\frac{1}{2}&-\frac{1}{2}&\frac{1}{2}} \\
   Y &= \myvec{1 & 0 & 2 & 0 & 0 \\
           0 & 1 & -1 & 0 & 0 \\
           0 & 0 & 0 & 1 & 0 \\
           0 & 0 & 0 & 0 & 1}
\end{align}
Now since the $Y^T$ matrix is full rank thus we can say that columns of matrix $A$ are linearly independent and b is described by \eqref{eq_2}
\end{document}